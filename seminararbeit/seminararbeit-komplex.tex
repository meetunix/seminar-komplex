\documentclass[a4paper,10pt]{scrreprt}
\usepackage[utf8]{inputenc}
\usepackage{minted}
\usepackage{zed-csp}
\usepackage{ngerman}
\usepackage{color}
\usepackage{colortbl}
\usepackage[dvipsnames]{xcolor}
\usepackage{fancyhdr}
\usepackage{lastpage}
\usepackage{geometry}
\usepackage{graphics}
\usepackage[pdftex]{graphicx}
\usepackage{hyperref}

\usepackage{setspace}
%\onehalfspacing      

\geometry{inner=2.5cm,outer=2.5cm,top=3cm,bottom=3cm}

\definecolor{shadecolor}{gray}{.92}

%%% Einrückabstand Absatz:
\parindent 0pt


%%% Kopf und Fußzeilen
\pagestyle{fancy}
\fancyhead{} % lösche Kopfzeile
\fancyhead[RO,LE]{Martin Steinbach}
\fancyhead[LO,RE]{Service und Security Monitoring}
\fancyfoot{} % lösche Fußzeile
\fancyfoot[LE,RO]{\thepage \hspace*{0.1cm} von\hspace*{0.2cm}\pageref{LastPage}}
\fancyfoot[LO,CE]{Seminar Aufbereitung und Auswertung komplexer Daten}
\fancyfoot[CO,RE]{}
\renewcommand{\headrulewidth}{0.4pt}
\renewcommand{\footrulewidth}{0.4pt}

\newenvironment{oneline}
{%begin code
\singlespacing \begin{it}
}
{%end code
\end{it} }

\newenvironment{onelinett}
{%begin code
\singlespacing \begin{tt}
}
{%end code
\end{tt} }

\newenvironment{onelinerm}
{%begin code
\singlespacing \begin{rm}
}
{%end code
\end{rm} }

\usepackage{hyperref}
\hypersetup{
	pdftitle={Service- and Security Monitoring },
	pdfauthor={Martin Steinbach},
	pdfkeywords={},
	pdfsubject={Seminar Aufbereitung und Auswertung komplexer Daten},
	pdfcreator={},
	citecolor=blue,
	hypertexnames=false,
	%linktocpage,
	pdfpagelabels,
	plainpages=false,
	backref,
	urlcolor=blue,
	menucolor=red,
	linkcolor=black,
	colorlinks=true,
	bookmarksnumbered,
	%pdffitwindow
}


%\renewcommand{\thesection}{\Roman{section}} 
%\renewcommand{\thesubsection}{\thesection.\Roman{subsection}}
\usemintedstyle{}
%opening
\title{Service and Security Monitoring}
\author{Martin Steinbach}

\date{\today}




\begin{document}

%\maketitle
%\thispagestyle{fancy}



\thispagestyle{empty}

\begin{center}
  
  {\Large \textsc{Seminararbeit}
  
  \vspace{4.25cm}
  
  {\fontsize{22}{22}\selectfont Service und Security-Monitoring\\}
  \vspace{0.75cm}
  {\fontsize{20}{20}\selectfont Seminar:\\ Aufbereitung und Auswertung komplexer Daten}
}
  
  \vspace{7.25cm}
  
  {\Large Martin Steinbach
    
    \vspace{.15cm}
    
    Juni 2018}
  
  \vspace{1.5cm}
  
  
  \includegraphics[scale=0.5]{img/siegel}
  
  \vspace{0.5cm}
  
  \rule{.7\textwidth}{.40pt}
  
  \vspace{.5cm}
  
  {\large\textsc{universität Rostock}}
    
    \vspace{.15cm}
        
\end{center}
\newpage 
\thispagestyle{empty}
\quad  \addtocounter{page}{-2}
\newpage
 
\vspace*{\fill}
\textbf{Exzerpt}\\\\
    Servicemonitoring ist eine wichtige Voraussetzung um eine zuverlässige 
    IT-Infrastruktur zu betreiben. Monitoring ist auch geeigent um IT-Sicherheitskritische
    Ereignisse zu identifizieren und adäquat auf diese zu reagieren. Die Vorliegende
    Arbeit bietet eine Einführung in die Thematik der Dienstüberwachung und stellt die
    beiden Überwachungsformen Aktive- und Passive-Überwachung vor. Es wird zudem
    die Frage geklärt, warum Servicemonitoring auch gleichzeitig Securitymonitoring ist.
    Anschließend wird
    anhand eines existierenden Prototypen aufgezeigt, wie eine Korrelation von 
    Ereignissen in Cloud-Umgebungen realisiert werden kann. In Abschnitt vier wird
    kurz auf eine schon verfügbare Lösungen im Bereich der Passiven-Überwachung 
    vorgestellt. 
\vspace*{\fill}


\tableofcontents
\thispagestyle{fancy}


%%%%%%%%%%%%%%%%%%%%%%%%%%%%%%%%%%%%%%%%%%%%%%%%%%%%%%%%%%%%%%%%%%%%%%%
% CHAPTER Einführung
%%%%%%%%%%%%%%%%%%%%%%%%%%%%%%%%%%%%%%%%%%%%%%%%%%%%%%%%%%%%%%%%%%%%%%%
\cleardoubleemptypage
\chapter{Einführung}
\thispagestyle{fancy}

In diesem Kapitel wird anhand der IT-Sicherheitsziele aufgezeigt, dass man unter dem
Servicemonitoring auch immer den Begriff Securitymonitoring verstehen
kann. Auch soll darauf hingewiesen werden, dass der Ausdruck Überwachung im ganzen
Dokument mit der Bedeutung: Aufsicht oder Monitoring belegt wird um eine klare Abgrenzung
zur zweiten Bedeutung: Observation, Beschattung (engl. surveillance) zu erlangen.


\section{Servicemonitoring und Securitymonitoring}

Die Überwachung von Diensten ist mittlerweile ein integraler Bestandteil der 
Infrastruktur jedes IT-~Diensteanbieters geworden. Neben der einfachen Erfassung
und der (z.B. grafischen) Aufarbeitung dieser Messgrößen, werden die erfassten Daten
zunehmend analysiert und es wird versucht Muster zu erkennen. Dieser Vorgang wird auch
als \textit{BigData} bezeichnet. Diese Daten werden auch verstärkt zur Sicherheitsanalyse
herangezogen. Daher stellt sich die Frage, ob Securitymonitoring äquivalent zum
Servicemonitoring-Begriff ist. Um es vorweg zunehmen, ja, denn es kommt ausschließlich auf
die Fragestellung an, die ich mit den erfasste Daten klären möchte. Im Folgenden werden 
die drei Hauptziele der IT-Sicherheit aufgeschlüsselt und in Beziehung mit dem 
Servicemonitoring gebracht.\\\\

\underline{\textbf{Vertraulichkeit}}\\\\
Das Ziel der Vertraulichkeit sagt aus, dass der Zugriff auf Daten ausschließlich von
autorisierten Nutzern erfolgen darf, egal in welchem Modus. Erreicht wird das Ziel zum
Beispiel durch Zugriffsrechte\footnote{Unabhängig von der Umsetzungsstrategie, wie z.B. 
Mandatory Access Contro (MAC) oder Discretionary Access Control(DAC)} und vor allem 
durch 
Verschlüsselung.\\

Die Frage, ob sich Vertraulichkeit überwachen lässt, ist nur teilweise beantwortbar.
Stellt man sich ein System vor auf dem ein nicht autorisierter Nutzer Zugriff auf
Informationen erlangt, so ist dies messbar und es ist möglich eine Meldung 
zu generieren (z.B. eine Log-Meldung oder eine Nachricht an Verantwortliche). Wird 
jedoch ein autorisiertes Konto durch einen nicht autorisierten Nutzer kompromittiert,
gestaltet sich die Entdeckung dieses Ereignisses schwieriger. Ob es sich in diesem Fall 
um einen erlaubten Zugriff des tatsächlichen Nutzers oder einen nicht erlaubten Zugriff
handelt kann nur unter der Zuhilfenahme weiterer Information geklärt werden,
zum Beispiel könnte die Quelle (Kapitel \ref{elk}), von der aus sich der Nutzer Zugriff 
verschafft hat,
miteinbezogen werden. Auch die Korrelation mit Zeitdaten, an denen sich der 
zugriffsberechtigte Nutzer einloggt, können zur Klärung hinzugezogen werden.\\\\

\underline{\textbf{Verfügbarkeit}}\\\\
Ob ein Dienst Verfügbar ist, wird dadurch geklärt, ob der Zugriff auf Informationen
innerhalb eines gewissen Zeitraums erfolgreich ist.\\

Die Verfügbarkeit gleicht damit auch der grundlegenden Fragestellung des 
Servicemonitorings. Ist ein gewisser Dienst erreichbar und ist dessen 
Abarbeitungsgeschwindigkeit in einem vorgegebenem Rahmen?\\\\

\newpage
\underline{\textbf{Integrität}}\\\\
Integrität wird erreicht, wenn eine Änderung der Daten nicht unbemerkt geschieht. Es soll 
somit ein Indikator für die Veränderung existieren. Um dieses Ziel zu erreichen werden
Verfahren wie digitale Signatur und Hashes verwendet.\\

Auch das Ziel der Integrität lässt sich kontrollieren, dazu finden die selben Maßnahmen 
Verwendung wie in der IT-Sicherheit. Es lassen sich zum Beispiel auf regelmäßiger Basis 
Daten prüfen, von denen man vorher mit einem kryptografisch sicheren Verfahren ein Hash 
errechnet hat. Ändert sich die Hashsumme, ohne das ein Zugriff auf die Daten genehmigt 
wurde, ist dies ein Integritätsverlust.
 
\section{Motive}

Der Grund warum eine dauerhafte Überwachung von Infrastruktur und den darauf aufbauenden 
Diensten keine Option sondern obligatorisch sein sollte, ist recht simpel zu erörtern. 
Allein die in \cite[461]{francia} berichteten Zahlen sprechen für sich. $90\,\%$ aller 
Firmen waren schon Cyberattacken ausgesetzt, $80\,\%$ davon haben dadurch erhebliche 
finanzielle Einbußen erlitten. Aktuell werden innerhalb eines Jahres $86\,\%$ der großen 
nordamerikanischen Unternehmen Opfer von Cyberattacken und der Diebstahl des geistigen 
Eigentums hat sich in den Jahren 2011-2015 verdoppelt.\\
Auch der aktuelle, jährlich veröffentlichte Lagebericht zur nationalen IT-Sicherheit des
Bundesamtes für Sicherheit in Informationstechnik (BSI) \cite[12]{bsi-lage}, berichtet  
von einer Cyberattacke auf eine großen deutschen Industriekonzern. Etwa zwei Monate 
konnten unbemerkt Daten aus weltweit verteilten Standorten in Richtung Südostasien 
abfließen bevor der Vorfall detektiert wurde. Aus den Empfehlungen des BSI lässt sich 
schließen, dass neben einer ungünstigen Netzwerksegmentierung auch mangelhaftes Monitoring
der Grund für die späte Erkennung war. 


\subsection{Gremien}

\section{Formen der Überwachung}
%%%%%%%%%%%%%%%%%%%%%%%%%%%%%%%%%%%%%%%%%%%%%%%%%%%%%%%%%%%%%%%%%%%%%%%
\chapter{Logkorrelation in Cloud-Umgebungen}\label{02_jcorrelat}
\thispagestyle{fancy}

Im folgenden Abschnitt wird eine Forschungsarbeit der Fachhochschule Fulda 
\cite{reissmann} vorgestellt. Ziel der Forschung war und ist es, ein gut skalierendes 
System zu entwickeln um eine automatisierte Auswertung von \textit{Syslog}-Meldungen in 
Cloud-Umgebungen bereit zu stellen. Aufgrund der enormen Datenmengen die in solchen 
Umgebungen anfallen kann eine Auswertung nur mittels korrelations- und 
Aggregationsverfahren geschehen. Um dieses Ziel zu erreichen kommen verschiedene 
Standards und eine Reihe von Software-Lösungen zum Einsatz.

Nachfolgend werden einige wichtige Begriffe geklärt, die Anforderungen identifiziert, die 
verwendeten Standards und die eingesetzte Software erläutert und im Weiteren Verlauf des 
Abschnitts wird anhand eines Beispiels eine \textit{syslog}-Korrelation vorgenommen.

\section{Anforderungen}\label{anforderungen}

Viele Unternehmen haben in den letzten Jahren einen großen Teil ihrer IT-Infrastruktur 
ausgelagert. Laut Analysen werden bis 2025 80 \% aller Unternehmen \cite{web_ix} 
weltweit ihre eigene Rechenzentrumsinfrastruktur abgeschaltet haben. Die wenigen Anbieter 
von Cloud-Infrastruktur stehen in Konkurrenz miteinander, daher existieren auch keine 
einheitlichen Schnittstellen um auf die Cloud-Konfigurationen zuzugreifen. Für die 
Überwachung, speziell von Sicherheitskritischen Ereignissen, stehen ebenso nur 
proprietäre Schnittstellen jedes Anbieters zur Verfügung. Aus diesem Grund soll ein 
System geschaffen werden, das die gewaltige Menge an aufkommenden Logdaten, unabhängig 
vom Anbieter, analysiert und sicherheitskritische Ereignisse unverzüglich zu 
identifiziert. Insbesondere soll das System das dynamisch sinkende und wachsende 
\textit{syslog}-Aufkommen beherrschen können. Denn durch die fluide Kostenstruktur der 
Cloud-Anbieter können schnell neue virtuelle Maschinen erstellt und entfernt werden, je 
nachdem wie viel Leistung der Kunde gerade benötigt.
Darüber hinaus sollen Meldungen auch persistent gespeichert werden, vornehmlich zur 
Erstellung von Trends und Langzeitanalysen, dabei soll der benötigte Speicherplatz so 
gering wie möglich gehalten werden.

\section{Beispielszenario}\label{szenario}

Im weiteren Verlauf dieses Kapitels soll zur detaillierteren Darstellung der 
Leistungsfähigkeit einer automatischen \textit{syslog}-Korrelation ein gängiges 
Angriffsszenario dienen. Eine \textit{ssh}-BruteForce Attacke auf eine beliebige Anzahl
an überwachten Systemen. Dabei soll genau der eine erfolgreiche Login innerhalb  der 
enormen Anzahl an erfolglosen oder ungültigen Versuchen identifiziert werden. Dieses 
recht einfache Szenario ist bei der erheblichen Anzahl an möglichen Systemen (10K+) 
manuell unmöglich zu bewerkstelligen.

\newpage
\section{JCorrelat}\label{sec_jcorrelat}

In Abbildung \ref{pic:jcorrelat} \cite[47]{reissmann} ist der schematische Aufbau von 
JCorrelat dargestellt, ein Prototyp der die Anforderungen aus Abschnitt 
\ref{anforderungen} erfüllen soll. Die hinzugefügten Nummern dienen der Übersichtlichkeit 
im weiteren Erklärungsverlauf.   

\begin{figure}[htbp]
    \caption{Aufbau von JCorrelat}
    \label{pic:jcorrelat}\vspace{0.2cm}
    \centering
    \includegraphics[scale=0.36]{img/schema-correlat}
    
\end{figure}

\subsection{Syslog-Protokoll} \label{syslog-proto}




\subsection{Konsolidierung von Syslog-Meldungen}\label{syslog-konsolidierung}
\subsection{persistente Speicherung}\label{nosql}
\subsection{Korrelation von Syslog-Meldungen}\label{syslog-korrelation}
%%%%%%%%%%%%%%%%%%%%%%%%%%%%%%%%%%%%%%%%%%%%%%%%%%%%%%%%%%%%%%%%%%%%%%%
\chapter{Weitere Lösungen zur passiven Überwachung}\label{elk}
\thispagestyle{fancy}
\include{03_elk}
%%%%%%%%%%%%%%%%%%%%%%%%%%%%%%%%%%%%%%%%%%%%%%%%%%%%%%%%%%%%%%%%%%%%%%%
\chapter{Ausblick}
\thispagestyle{fancy}
\include{04_ausbl}
%%%%%%%%%%%%%%%%%%%%%%%%%%%%%%%%%%%%%%%%%%%%%%%%%%%%%%%%%%%%%%%%%%%%%%%
\chapter{Fazit}
\thispagestyle{fancy}
\include{05_fazit}
%%%%%%%%%%%%%%%%%%%%%%%%%%%%%%%%%%%%%%%%%%%%%%%%%%%%%%%%%%%%%%%%%%%%%%%
\bibliography{bib/bibo.bib}{}
\bibliographystyle{unsrt}
\thispagestyle{fancy}

\listoffigures
\thispagestyle{fancy}


\end{document}
