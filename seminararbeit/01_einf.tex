\chapter{Einführung}
\thispagestyle{fancy}

In diesem Kapitel wird anhand der IT-Sicherheitsziele aufgezeigt, dass man unter dem
Servicemonitoring auch immer den Begriff Securitymonitoring verstehen
kann. Auch soll darauf hingewiesen werden, dass der Ausdruck Überwachung im ganzen
Dokument mit der Bedeutung: Aufsicht oder Monitoring belegt wird um eine klare Abgrenzung
zur zweiten Bedeutung: Observation, Beschattung (engl. surveillance) zu erlangen.


\section{Servicemonitoring und Securitymonitoring}

Die Überwachung von Diensten ist mittlerweile ein integraler Bestandteil der 
Infrastruktur jedes IT-~Diensteanbieters geworden. Neben der einfachen Erfassung
und der (z.B. grafischen) Aufarbeitung dieser Messgrößen, werden die erfassten Daten
zunehmend analysiert und es wird versucht Muster zu erkennen. Dieser Vorgang wird auch
als \textit{BigData} bezeichnet. Diese Daten werden auch verstärkt zur Sicherheitsanalyse
herangezogen. Daher stellt sich die Frage, ob Securitymonitoring äquivalent zum
Servicemonitoring-Begriff ist. Um es vorweg zunehmen, ja, denn es kommt ausschließlich auf
die Fragestellung an, die ich mit den erfasste Daten klären möchte. Im Folgenden werden 
die drei Hauptziele der IT-Sicherheit aufgeschlüsselt und in Beziehung mit dem 
Servicemonitoring gebracht.\\\\

\underline{\textbf{Vertraulichkeit}}\\\\
Das Ziel der Vertraulichkeit sagt aus, dass der Zugriff auf Daten ausschließlich von
autorisierten Nutzern erfolgen darf, egal in welchem Modus. Erreicht wird das Ziel zum
Beispiel durch Zugriffsrechte\footnote{Unabhängig von der Umsetzungsstrategie, wie z.B. 
Mandatory Access Contro (MAC) oder Discretionary Access Control(DAC)} und vor allem 
durch 
Verschlüsselung.\\

Die Frage, ob sich Vertraulichkeit überwachen lässt, ist nur teilweise beantwortbar.
Stellt man sich ein System vor auf dem ein nicht autorisierter Nutzer Zugriff auf
Informationen erlangt, so ist dies messbar und es ist möglich eine Meldung 
zu generieren (z.B. eine Log-Meldung oder eine Nachricht an Verantwortliche). Wird 
jedoch ein autorisiertes Konto durch einen nicht autorisierten Nutzer kompromittiert,
gestaltet sich die Entdeckung dieses Ereignisses schwieriger. Ob es sich in diesem Fall 
um einen erlaubten Zugriff des tatsächlichen Nutzers oder einen nicht erlaubten Zugriff
handelt kann nur unter der Zuhilfenahme weiterer Information geklärt werden,
zum Beispiel könnte die Quelle (Kapitel \ref{elk}), von der aus sich der Nutzer Zugriff 
verschafft hat,
miteinbezogen werden. Auch die Korrelation mit Zeitdaten, an denen sich der 
zugriffsberechtigte Nutzer einloggt, können zur Klärung hinzugezogen werden.\\\\

\underline{\textbf{Verfügbarkeit}}\\\\
Ob ein Dienst Verfügbar ist, wird dadurch geklärt, ob der Zugriff auf Informationen
innerhalb eines gewissen Zeitraums erfolgreich ist.\\

Die Verfügbarkeit gleicht damit auch der grundlegenden Fragestellung des 
Servicemonitorings. Ist ein gewisser Dienst erreichbar und ist dessen 
Abarbeitungsgeschwindigkeit in einem vorgegebenem Rahmen?\\\\

\newpage
\underline{\textbf{Integrität}}\\\\
Integrität wird erreicht, wenn eine Änderung der Daten nicht unbemerkt geschieht. Es soll 
somit ein Indikator für die Veränderung existieren. Um dieses Ziel zu erreichen werden
Verfahren wie digitale Signatur und Hashes verwendet.\\

Auch das Ziel der Integrität lässt sich kontrollieren, dazu finden die selben Maßnahmen 
Verwendung wie in der IT-Sicherheit. Es lassen sich zum Beispiel auf regelmäßiger Basis 
Daten prüfen, von denen man vorher mit einem kryptografisch sicheren Verfahren ein Hash 
errechnet hat. Ändert sich die Hashsumme, ohne das ein Zugriff auf die Daten genehmigt 
wurde, ist dies ein Integritätsverlust.
 
\section{Motive}

Der Grund warum eine dauerhafte Überwachung von Infrastruktur und den darauf aufbauenden 
Diensten keine Option sondern obligatorisch sein sollte, ist recht simpel zu erörtern. 
Allein die in \cite[461]{francia} berichteten Zahlen sprechen für sich. $90\,\%$ aller 
Firmen waren schon Cyberattacken ausgesetzt, $80\,\%$ davon haben dadurch erhebliche 
finanzielle Einbußen erlitten. Aktuell werden innerhalb eines Jahres $86\,\%$ der großen 
nordamerikanischen Unternehmen Opfer von Cyberattacken und der Diebstahl des geistigen 
Eigentums hat sich in den Jahren 2011-2015 verdoppelt.\\
Auch der aktuelle, jährlich veröffentlichte Lagebericht zur nationalen IT-Sicherheit des
Bundesamtes für Sicherheit in Informationstechnik (BSI) \cite[12]{bsi-lage}, berichtet  
von einer Cyberattacke auf eine großen deutschen Industriekonzern. Etwa zwei Monate 
konnten unbemerkt Daten aus weltweit verteilten Standorten in Richtung Südostasien 
abfließen bevor der Vorfall detektiert wurde. Aus den Empfehlungen des BSI lässt sich 
schließen, dass neben einer ungünstigen Netzwerksegmentierung auch mangelhaftes Monitoring
der Grund für die späte Erkennung war. 


\subsection{Gremien}

\section{Formen der Überwachung}