\chapter{Einführung}\label{01_einf}
\thispagestyle{fancy}

In diesem Kapitel wird anhand der IT-Sicherheitsziele aufgezeigt, dass man unter dem
Servicemonitoring auch immer den Begriff Securitymonitoring verstehen
kann. Auch soll darauf hingewiesen werden, dass der Ausdruck Überwachung im ganzen
Dokument mit der Bedeutung: Aufsicht oder Monitoring belegt wird um eine klare Abgrenzung
zur zweiten Bedeutung: Observation, Beschattung (engl. surveillance) zu erlangen.


\section{Servicemonitoring und Securitymonitoring}

Die Überwachung von Diensten ist mittlerweile ein integraler Bestandteil der 
Infrastruktur jedes IT-~Diensteanbieters geworden. Neben der einfachen Erfassung
und der (z.B. grafischen) Aufarbeitung dieser Messgrößen, werden die erfassten Daten
zunehmend analysiert und es wird versucht Muster zu erkennen. Dieser Vorgang wird auch
als \textit{BigData} bezeichnet. Diese Daten werden auch verstärkt zur Sicherheitsanalyse
herangezogen. Daher stellt sich die Frage, ob Securitymonitoring äquivalent zum
Servicemonitoring-Begriff ist. Um es vorweg zunehmen, ja, denn es kommt ausschließlich auf
die Fragestellung an, die ich mit den erfasste Daten klären möchte. Im Folgenden werden 
die drei Hauptziele der IT-Sicherheit aufgeschlüsselt und in Beziehung mit dem 
Servicemonitoring gebracht.\\\\

\underline{\textbf{Vertraulichkeit}}\\\\
Das Ziel der Vertraulichkeit sagt aus, dass der Zugriff auf Daten ausschließlich von
autorisierten Nutzern erfolgen darf, egal in welchem Modus. Erreicht wird das Ziel zum
Beispiel durch Zugriffsrechte\footnote{Unabhängig von der Umsetzungsstrategie, wie z.B. 
Mandatory Access Contro (MAC) oder Discretionary Access Control(DAC)} und vor allem 
durch 
Verschlüsselung.\\

Die Frage, ob sich Vertraulichkeit überwachen lässt, ist nur teilweise beantwortbar.
Stellt man sich ein System vor auf dem ein nicht autorisierter Nutzer Zugriff auf
Informationen erlangt, so ist dies messbar und es ist möglich eine Meldung 
zu generieren (z.B. eine Log-Meldung oder eine Nachricht an Verantwortliche). Wird 
jedoch ein autorisiertes Konto durch einen nicht autorisierten Nutzer kompromittiert,
gestaltet sich die Entdeckung dieses Ereignisses schwieriger. Ob es sich in diesem Fall 
um einen erlaubten Zugriff des tatsächlichen Nutzers oder einen nicht erlaubten Zugriff
handelt kann nur unter der Zuhilfenahme weiterer Information geklärt werden,
zum Beispiel könnte die Quelle (Kapitel \ref{elk}), von der aus sich der Nutzer Zugriff 
verschafft hat,
miteinbezogen werden. Auch die Korrelation mit Zeitdaten, an denen sich der 
zugriffsberechtigte Nutzer einloggt, können zur Klärung hinzugezogen werden.\\\\

\underline{\textbf{Verfügbarkeit}}\\\\
Ob ein Dienst Verfügbar ist, wird dadurch geklärt, ob der Zugriff auf Informationen
innerhalb eines gewissen Zeitraums erfolgreich ist.\\

Die Verfügbarkeit gleicht damit auch der grundlegenden Fragestellung des 
Servicemonitorings. Ist ein gewisser Dienst erreichbar und ist dessen 
Abarbeitungsgeschwindigkeit in einem vorgegebenem Rahmen?\\\\
\newpage
\underline{\textbf{Integrität}}\\\\
Integrität wird erreicht, wenn eine Änderung der Daten nicht unbemerkt geschieht. Es soll 
somit ein Indikator für die Veränderung existieren. Um dieses Ziel zu erreichen werden
Verfahren wie digitale Signatur und Hashes verwendet.\\

Auch das Ziel der Integrität lässt sich kontrollieren, dazu finden die selben Maßnahmen 
Verwendung wie in der IT-Sicherheit. Es lassen sich zum Beispiel auf regelmäßiger Basis 
Daten prüfen, von denen man vorher mit einem kryptografisch sicheren Verfahren ein Hash 
errechnet hat. Ändert sich die Hashsumme, ohne das ein Zugriff auf die Daten genehmigt 
wurde, ist dies ein Integritätsverlust.\\

Zusammengefasst ist Monitoring von IT-Infrastruktur immer auch gleich Securitymonitoring. 
Mithilfe eines lückenlos ausgerollten Monitorings ist es demnach möglich zu Beweisen zu 
welchem Zeitpunkt ein gewisser Dienst welchen Zustand hatte.
 
\section{Motive}

Der Grund warum eine dauerhafte Überwachung von Infrastruktur und den darauf aufbauenden 
Diensten keine Option sondern obligatorisch sein sollte, ist recht simpel zu erörtern. 
Allein die in \cite[461]{francia} berichteten Zahlen sprechen für sich. $90\,\%$ aller 
Firmen waren schon Cyberattacken ausgesetzt, $80\,\%$ davon haben dadurch erhebliche 
finanzielle Einbußen erlitten. Aktuell werden innerhalb eines Jahres $86\,\%$ der großen 
nordamerikanischen Unternehmen Opfer von Cyberattacken und der Diebstahl des geistigen 
Eigentums hat sich in den Jahren 2011-2015 verdoppelt.\\
Auch der aktuelle, jährlich veröffentlichte Lagebericht zur nationalen IT-Sicherheit des
Bundesamtes für Sicherheit in Informationstechnik (BSI) \cite[12]{bsi-lagebericht} 
berichtet  
von einer Cyberattacke auf eine großen deutschen Industriekonzern. Etwa zwei Monate 
konnten unbemerkt Daten aus weltweit verteilten Standorten in Richtung Südostasien 
abfließen bevor der Vorfall detektiert wurde. Aus den Empfehlungen des BSI lässt sich 
schließen, dass neben einer ungünstigen Netzwerksegmentierung auch mangelhaftes Monitoring
der Grund für die späte Erkennung war. In diesem Zusammenhang ist auch der Angriff auf 
den Deutschen Bundestag im Jahr 2015 erwähnenswert, da auch in diesem Fall einige Wochen 
lang unbemerkt Daten abfließen konnten und das Ziel hoheitliches war, 
Technologietransfer und finanzielle Absichten also eine untergeordnete Rolle gespielt 
haben.
%\newpage
\subsection{Behörden mit Überwachungsauftrag}

Aufgrund der zuvor dargestellten Gründe, wurden in den letzten Jahren eine Reihe an neuen 
Behörden in der Bundesrepublik Deutschland gegründet, deren Auftrag die Überwachung 
(Monitoring) wichtiger Infrastrukturen innerhalb der Grenzen der Bundesrepublik ist. Zum 
Einen das Nationale Cyber-Abwehrzentrum (NCAZ) \cite{web_ncaz} mit Sitz in Bonn. Die 
Aufgabe 
des NCAZ ist 
die Koordinierung von Abwehrmaßnahmen und Informationskonsolidierung über den Aufbau von 
rein ziviler Infrastruktur über mehrere Behörden hinweg. Das Nationale IT-Lagezentrum 
\cite{web_lagezentrum} 
Überwacht hingegen aktiv die Regierungsnetze und erstellt monatliche Lageberichte. Auf 
militärischer Seite übernimmt der Bundeswehr-Organisationsbereich Cyber- und 
Informationsraum (CIR) diese Aufgabe.

\section{Überwachungsformen}

Es lassen sich zwei verschiedene Überwachungsformen identifizieren. Die aktive 
Überwachung, bei der aktiv Status und Messwerte von Diensten erfragt werden, sowie die 
passive Überwachung, bei der die Dienste selbstständig Meldungen an einen zentralen Punkt
senden. Unter einem Dienst wird in diesen Beispielen nicht nur ein \textit{service oder 
daemon} sondern ein jegliche Entität, deren Status Messbar ist, verstanden.

\subsection{Aktive Überwachung}
\begin{figure}[htbp]
    \caption{Sequenzdiagramm: Aktive Überwachung}
    \label{aktiv}\vspace{0.2cm}
    \centering
    \includegraphics[scale=0.36]{img/sequence_uml_active_trans}

\end{figure}

Obiges Sequenzdiagramm zeigt schematisch den Ablauf aktiver Überwachung. \emph{Monitor} 
ist die zentrale Instanz auf der alle zu überwachenden Informationen gesammelt, 
aufbereitet und ausgegeben werden. Jedoch muss der \emph{Monitor} nicht alle Information
sammeln, es können auch hierarchisch untergeordnete \emph{Aggregatoren} existieren, 
welche ebenfalls Informationen von verschiedenen Diensten erheben. Diese 
\emph{Aggregatoren} können aus Leistungsgründen vor einen \emph{Monitor} geschaltet 
werden, um die Anzahl an abzufragenden Diensten für den \emph{Monitor} zu verringern. In 
diesem Fall bereitet bereits der \emph{Aggregator} die Daten auf und der \emph{Monitor} 
fragt nur noch die schon konsolidierten Informationen ab. Aber auch Segmentierungen von 
Infrastruktur können diesen Aufbau notwendig machen, wenn z.B. \emph{Monitor} 
\emph{Dienst A} und \emph{Dienst B} nicht direkt erreichen kann oder darf. Zur 
Klassifizierung von Ereignissen werden in Überwachungslösungen\footnote{zum Beispiel: 
Nagios, Icinga} oft verschiedene Status verwendet, dies dient hauptsächlich zum 
schnelleren Verständnis für die auswertende Person. Aus diesem Grund wurden in 
Abbildung \ref{aktiv} die Bezeichnungen für der Abfragefunktionen \texttt{getState()} und 
\texttt{reState()} gewählt. Mögliche Status für die Rückgabe sind in Tabelle 
\ref{table:status} aufgeführt.

\begin{table}[ht]
    \caption{Statusübersicht}
    \label{table:status}\vspace{0.2cm}
    \centering{
    \renewcommand{\arraystretch}{1.3}
    
    \begin{tabular}{|l|l|}
        
        \hline
       \rowcolor{gray!40} \textbf{Statusbezeichnung} & \textbf{Statusbeschreibung}\\
        \hline
        \texttt{OK} & Dienst läuft innerhalb normaler Parameter.\\
        \texttt{WARNING}& Die (zuvor definierte) Warnschwelle wurde überschritten.\\
        \texttt{CRITICAL}& Die kritische Schwelle wurde überschritten oder es gab einen
        Timeout.\\
        \texttt{UNKNOWN}& Ein undefinierter Wert wurde an \emph{Monitor} übermittelt.\\
        \hline
    \end{tabular}
}

\end{table}
\newpage
Mithilfe der Techniken zur aktiven Überwachung lassen sich zwei Klassen überwachbarer 
Dienste identifizieren: Die Betriebssystemabhängigen und die Betriebssystemunabhängigen 
Dienste. Tabelle \ref{table:aktiv} listet einige Beispiele für die jeweilige Klasse auf.


     
\begin{table}[ht]
\caption{Beispiele für aktive Überwachung}
\label{table:aktiv}\vspace{0.2cm}
\centering{
\renewcommand{\arraystretch}{1.3}
\begin{tabular}{|l|l|}
    \hline
    \rowcolor{gray!40}\textbf{Dienst / Entität }& \textbf{Beispiel} \\
    \hline
    \multicolumn{2}{|l|}{\cellcolor{shadecolor}\textbf{Betriebssystemabhängig}}\\
    \hline
    Auslastung & wie viel CPU-Zeit benötigt ein bestimmter Prozess/das ganze System\\
    Speicher & Speicherauslastung des Systems/ Belegung persistenter Speicher\\
    Prozesse & läuft ein bestimmter Prozess/ wie viele Prozesse eines Namens laufen\\
    Datendurchsatz & wie viele Bytes passieren ein Interface, Anzahl an
    \emph{paket-drops},\emph{rejects}\\
    Audit & wurden Zugriffsregeln verletzt/ welcher Nutzer hat auf Datei X zugegriffen\\
    \hline
    \multicolumn{2}{|l|}{\cellcolor{shadecolor}\textbf{Betriebssystemunabhängig}}\\
    \hline
    \texttt{ICMP} & Netzwerkschnittstelle/ System erreichbar, \emph{round trip 
    time}\\
    \texttt{TCP/UDP} & ist bestimmter \emph{port} erreichbar\\
    Anwendungsprotokolle & Login möglich / Referenzdaten abrufbar / Rückgabewerte 
    Testroutinen\\
    \texttt{SNMP} & Abfrage beliebiger Kenngrößen\\
    \hline
\end{tabular}


}
\end{table}

\subsection{Passive Überwachung}
\begin{figure}[htbp]
    \caption{Passive Überwachung}
    \label{passiv}\vspace{0.2cm}
    \centering
    \includegraphics[scale=0.36]{img/sequence_uml_passive_trans}

\end{figure}

Wie in Abbildung \ref{passiv} zu erkennen, werden bei der passiven Überwachung nur Daten
ausgewertet, welche durch Dienste selbst generiert werden oder aber durch eine Software, 
welche den Dienst lokal überwacht. Es erfolgt keine Abfrage bei den Diensten, der 
zentrale Punkt, der die Auswertung übernimmt bleibt passiv und empfängt lediglich 
Meldungen. Am häufigsten sind diese Meldungen LOG-Meldungen, generiert von einem 
LOG-System. Aber auch SNMP-Traps fallen unter diese Kategorie, daher auch die Wahl der 
Funktionen in Abbildung \ref{passiv}.