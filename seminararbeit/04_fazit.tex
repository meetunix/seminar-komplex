\chapter{Fazit und Ausblick}
\thispagestyle{fancy}

Mit \textit{JCorrelat} wurde ein Prototyp für die automatische Korrelation und 
Konsolidierung von Syslog-Meldungen aus unterschiedlichen Quellen geschaffen. Wie die 
Ergebnisse des Benchmark zeigen skaliert der Prototyp sehr gut, wenn für jede Anwendung 
eine \textit{JCorrelat}-Instanz verwendet wird. Weniger Verwaltungsaufwand würde 
allerdings die Verwendung einer (eventuell auch proprietären) 
\textit{distributet-memory}-Lösung bieten, da dann die \textit{Drools-Fusion}-Regeln 
global platziert werden könnten und jede \textit{Drools}-Instanz auf den selben Daten 
operieren könnte. Arbeitsspeicher ist der limitierende Faktor des Korrelationssystems, je 
mehr Arbeitsspeicher zur Verfügung steht umso mehr und größere Betrachtungsfenster kann 
\textit{Drools-Fusion} verwenden und so bessere Ergebnisse erzielen.\\
Wie fortlaufend am Beispiel aus Sektion \ref{szenario} gezeigt, ist der Prototyp in der 
Lage sicherheitskritische Ereignisse sicher zu identifizieren und eignet sich so bestens 
für die Überwachung von großen IT-Umgebungen. Anzumerken ist allerdings, dass die 
Zeitfenster in den Beispielregeln (Anhang: Abbildung \ref{app:drools-warning} und 
Abbildung \ref{app:drools-emergency}) recht willkürlich gewählt wurden und in 
Produktivumgebungen sicherlich angepasst werden müssten.\\

Durch die Normalisierung der Syslog-Meldung wird eine sehr effiziente Verdichtung von 
Informationen erreicht und Redundanzen werden größtenteils eliminiert. Das führt neben 
der damit verbundenen hohen Ausführungsgeschwindigkeit der Korrelation auch zu einem 
stark reduzierten Speicherbedarf für die persistente Speicherung. Durch die Wahl, die 
normalisierten und korrelierten Daten in \textit{elasticsearch} abzuspeichern, bieten 
sich vielfältige Möglichkeiten der Auswertung und Visualisierung an. So Könnten bereits 
bestehende Monitoringsysteme um Module erweitert werden, welche die zuvor erstellten 
Ereignisse periodisch abfragen und somit alarmieren können.\\

Auch eine Korrelation von weiteren Daten ist realisierbar, so könnten die 
Syslog-Meldungen zum Beispiel mit Geo- oder Performance-Informationen angereichert 
werden. Falls nicht genügend Arbeitsspeicher zur Verfügung steht, aber eine hohe 
Korrelationsgenauigkeit erreicht werden soll und die Korrelationsgeschwindigkeit keine 
Rolle spielt, könnte \textit{JCorrelat} auch auf den Datenbestand in 
\textit{elasticsearch} zugreifen.\\
Um nicht alle Regeln (je Dienst) einzeln konfigurieren zu müssen, soll das Ziel sein, 
\textit{JCorrelat} mit Hilfe von Vorlagen konfigurieren zu können. Damit könnten neu 
hinzukommende Dienste einer automatischen Korrelation unterzogen werden. Darüber hinaus 
wäre auch eine Integration in \textit{kubernetes}\footnote{\url{https://kubernetes.io/}}, 
\textit{openstack}\footnote{\url{https://www.openstack.org/}} oder dem ELK-Stack 
(Abschnitt \ref{ausblick}) erstrebenswert. 