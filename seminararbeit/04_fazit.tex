\chapter{Fazit und Ausblick}
\thispagestyle{fancy}

Mit \textit{JCorrelat} wurde ein Prototyp für die automatische Korrelation und 
Konsolidierung von Syslog-Meldungen aus unterschiedlichen Quellen geschaffen. Wie die 
Ergebnisse des Benchmark zeigen skaliert der Prototyp sehr gut, wenn für jede Anwendung 
eine \textit{JCorrelat}-Instanz verwendet wird. Weniger Verwaltungsaufwand würde 
allerdings die Verwendung einer (eventuell auch proprietären) 
\textit{distributet-memory}-Lösung bieten, da dann die \textit{Drools-Fusion}-Regeln 
global platziert werden könnten und jede \textit{Drools}-Instanz auf den selben Daten 
arbeiten kann. Arbeitsspeicher ist auch der limitierende Faktor des Korrelationssystems, 
je mehr Arbeitsspeicher zur Verfügung steht umso mehr und größere Betrachtungsfenster 
kann \textit{Drools-Fusion} verwenden und so bessere Ergebnisse erzielen.\\

Durch die Normalisierung der Syslog-Meldung wird eine sehr effiziente Verdichtung von 
Informationen erreicht und Redundanzen werden größtenteils eliminiert. Das führt, neben 
der damit verbundenen hohen Ausführungsgeschwindigkeit der Korrelation, auch zu einem 
stark reduzierten Speicherbedarf für die persistente Speicherung. Durch die Wahl die 
normalisierten und korrelierten Daten in \textit{elasticsearch} abzuspeichern, bieten 
sich vielfältige Möglichkeiten der Auswertung und Visualisierung an. So Könnten bereits 
bestehende Monitoringsysteme um Plugins erweitert werden, welche die durch 
\textit{JCorrelat} erstellten Ereignisse periodisch abfragen und alarmieren können.\\

Auch eine Korrelation von weiteren Daten ist realisierbar, so könnten die 
Syslog-Meldungen auch mit Geo-Daten angereichert werde. 